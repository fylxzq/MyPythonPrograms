
% Default to the notebook output style

    


% Inherit from the specified cell style.




    
\documentclass[11pt]{article}

    
    
    \usepackage[T1]{fontenc}
    % Nicer default font (+ math font) than Computer Modern for most use cases
    \usepackage{mathpazo}

    % Basic figure setup, for now with no caption control since it's done
    % automatically by Pandoc (which extracts ![](path) syntax from Markdown).
    \usepackage{graphicx}
    % We will generate all images so they have a width \maxwidth. This means
    % that they will get their normal width if they fit onto the page, but
    % are scaled down if they would overflow the margins.
    \makeatletter
    \def\maxwidth{\ifdim\Gin@nat@width>\linewidth\linewidth
    \else\Gin@nat@width\fi}
    \makeatother
    \let\Oldincludegraphics\includegraphics
    % Set max figure width to be 80% of text width, for now hardcoded.
    \renewcommand{\includegraphics}[1]{\Oldincludegraphics[width=.8\maxwidth]{#1}}
    % Ensure that by default, figures have no caption (until we provide a
    % proper Figure object with a Caption API and a way to capture that
    % in the conversion process - todo).
    \usepackage{caption}
    \DeclareCaptionLabelFormat{nolabel}{}
    \captionsetup{labelformat=nolabel}

    \usepackage{adjustbox} % Used to constrain images to a maximum size 
    \usepackage{xcolor} % Allow colors to be defined
    \usepackage{enumerate} % Needed for markdown enumerations to work
    \usepackage{geometry} % Used to adjust the document margins
    \usepackage{amsmath} % Equations
    \usepackage{amssymb} % Equations
    \usepackage{textcomp} % defines textquotesingle
    % Hack from http://tex.stackexchange.com/a/47451/13684:
    \AtBeginDocument{%
        \def\PYZsq{\textquotesingle}% Upright quotes in Pygmentized code
    }
    \usepackage{upquote} % Upright quotes for verbatim code
    \usepackage{eurosym} % defines \euro
    \usepackage[mathletters]{ucs} % Extended unicode (utf-8) support
    \usepackage[utf8x]{inputenc} % Allow utf-8 characters in the tex document
    \usepackage{fancyvrb} % verbatim replacement that allows latex
    \usepackage{grffile} % extends the file name processing of package graphics 
                         % to support a larger range 
    % The hyperref package gives us a pdf with properly built
    % internal navigation ('pdf bookmarks' for the table of contents,
    % internal cross-reference links, web links for URLs, etc.)
    \usepackage{hyperref}
    \usepackage{longtable} % longtable support required by pandoc >1.10
    \usepackage{booktabs}  % table support for pandoc > 1.12.2
    \usepackage[inline]{enumitem} % IRkernel/repr support (it uses the enumerate* environment)
    \usepackage[normalem]{ulem} % ulem is needed to support strikethroughs (\sout)
                                % normalem makes italics be italics, not underlines
    

    
    
    % Colors for the hyperref package
    \definecolor{urlcolor}{rgb}{0,.145,.698}
    \definecolor{linkcolor}{rgb}{.71,0.21,0.01}
    \definecolor{citecolor}{rgb}{.12,.54,.11}

    % ANSI colors
    \definecolor{ansi-black}{HTML}{3E424D}
    \definecolor{ansi-black-intense}{HTML}{282C36}
    \definecolor{ansi-red}{HTML}{E75C58}
    \definecolor{ansi-red-intense}{HTML}{B22B31}
    \definecolor{ansi-green}{HTML}{00A250}
    \definecolor{ansi-green-intense}{HTML}{007427}
    \definecolor{ansi-yellow}{HTML}{DDB62B}
    \definecolor{ansi-yellow-intense}{HTML}{B27D12}
    \definecolor{ansi-blue}{HTML}{208FFB}
    \definecolor{ansi-blue-intense}{HTML}{0065CA}
    \definecolor{ansi-magenta}{HTML}{D160C4}
    \definecolor{ansi-magenta-intense}{HTML}{A03196}
    \definecolor{ansi-cyan}{HTML}{60C6C8}
    \definecolor{ansi-cyan-intense}{HTML}{258F8F}
    \definecolor{ansi-white}{HTML}{C5C1B4}
    \definecolor{ansi-white-intense}{HTML}{A1A6B2}

    % commands and environments needed by pandoc snippets
    % extracted from the output of `pandoc -s`
    \providecommand{\tightlist}{%
      \setlength{\itemsep}{0pt}\setlength{\parskip}{0pt}}
    \DefineVerbatimEnvironment{Highlighting}{Verbatim}{commandchars=\\\{\}}
    % Add ',fontsize=\small' for more characters per line
    \newenvironment{Shaded}{}{}
    \newcommand{\KeywordTok}[1]{\textcolor[rgb]{0.00,0.44,0.13}{\textbf{{#1}}}}
    \newcommand{\DataTypeTok}[1]{\textcolor[rgb]{0.56,0.13,0.00}{{#1}}}
    \newcommand{\DecValTok}[1]{\textcolor[rgb]{0.25,0.63,0.44}{{#1}}}
    \newcommand{\BaseNTok}[1]{\textcolor[rgb]{0.25,0.63,0.44}{{#1}}}
    \newcommand{\FloatTok}[1]{\textcolor[rgb]{0.25,0.63,0.44}{{#1}}}
    \newcommand{\CharTok}[1]{\textcolor[rgb]{0.25,0.44,0.63}{{#1}}}
    \newcommand{\StringTok}[1]{\textcolor[rgb]{0.25,0.44,0.63}{{#1}}}
    \newcommand{\CommentTok}[1]{\textcolor[rgb]{0.38,0.63,0.69}{\textit{{#1}}}}
    \newcommand{\OtherTok}[1]{\textcolor[rgb]{0.00,0.44,0.13}{{#1}}}
    \newcommand{\AlertTok}[1]{\textcolor[rgb]{1.00,0.00,0.00}{\textbf{{#1}}}}
    \newcommand{\FunctionTok}[1]{\textcolor[rgb]{0.02,0.16,0.49}{{#1}}}
    \newcommand{\RegionMarkerTok}[1]{{#1}}
    \newcommand{\ErrorTok}[1]{\textcolor[rgb]{1.00,0.00,0.00}{\textbf{{#1}}}}
    \newcommand{\NormalTok}[1]{{#1}}
    
    % Additional commands for more recent versions of Pandoc
    \newcommand{\ConstantTok}[1]{\textcolor[rgb]{0.53,0.00,0.00}{{#1}}}
    \newcommand{\SpecialCharTok}[1]{\textcolor[rgb]{0.25,0.44,0.63}{{#1}}}
    \newcommand{\VerbatimStringTok}[1]{\textcolor[rgb]{0.25,0.44,0.63}{{#1}}}
    \newcommand{\SpecialStringTok}[1]{\textcolor[rgb]{0.73,0.40,0.53}{{#1}}}
    \newcommand{\ImportTok}[1]{{#1}}
    \newcommand{\DocumentationTok}[1]{\textcolor[rgb]{0.73,0.13,0.13}{\textit{{#1}}}}
    \newcommand{\AnnotationTok}[1]{\textcolor[rgb]{0.38,0.63,0.69}{\textbf{\textit{{#1}}}}}
    \newcommand{\CommentVarTok}[1]{\textcolor[rgb]{0.38,0.63,0.69}{\textbf{\textit{{#1}}}}}
    \newcommand{\VariableTok}[1]{\textcolor[rgb]{0.10,0.09,0.49}{{#1}}}
    \newcommand{\ControlFlowTok}[1]{\textcolor[rgb]{0.00,0.44,0.13}{\textbf{{#1}}}}
    \newcommand{\OperatorTok}[1]{\textcolor[rgb]{0.40,0.40,0.40}{{#1}}}
    \newcommand{\BuiltInTok}[1]{{#1}}
    \newcommand{\ExtensionTok}[1]{{#1}}
    \newcommand{\PreprocessorTok}[1]{\textcolor[rgb]{0.74,0.48,0.00}{{#1}}}
    \newcommand{\AttributeTok}[1]{\textcolor[rgb]{0.49,0.56,0.16}{{#1}}}
    \newcommand{\InformationTok}[1]{\textcolor[rgb]{0.38,0.63,0.69}{\textbf{\textit{{#1}}}}}
    \newcommand{\WarningTok}[1]{\textcolor[rgb]{0.38,0.63,0.69}{\textbf{\textit{{#1}}}}}
    
    
    % Define a nice break command that doesn't care if a line doesn't already
    % exist.
    \def\br{\hspace*{\fill} \\* }
    % Math Jax compatability definitions
    \def\gt{>}
    \def\lt{<}
    % Document parameters
    \title{2.matplotlib?????}
    
    
    

    % Pygments definitions
    
\makeatletter
\def\PY@reset{\let\PY@it=\relax \let\PY@bf=\relax%
    \let\PY@ul=\relax \let\PY@tc=\relax%
    \let\PY@bc=\relax \let\PY@ff=\relax}
\def\PY@tok#1{\csname PY@tok@#1\endcsname}
\def\PY@toks#1+{\ifx\relax#1\empty\else%
    \PY@tok{#1}\expandafter\PY@toks\fi}
\def\PY@do#1{\PY@bc{\PY@tc{\PY@ul{%
    \PY@it{\PY@bf{\PY@ff{#1}}}}}}}
\def\PY#1#2{\PY@reset\PY@toks#1+\relax+\PY@do{#2}}

\expandafter\def\csname PY@tok@w\endcsname{\def\PY@tc##1{\textcolor[rgb]{0.73,0.73,0.73}{##1}}}
\expandafter\def\csname PY@tok@c\endcsname{\let\PY@it=\textit\def\PY@tc##1{\textcolor[rgb]{0.25,0.50,0.50}{##1}}}
\expandafter\def\csname PY@tok@cp\endcsname{\def\PY@tc##1{\textcolor[rgb]{0.74,0.48,0.00}{##1}}}
\expandafter\def\csname PY@tok@k\endcsname{\let\PY@bf=\textbf\def\PY@tc##1{\textcolor[rgb]{0.00,0.50,0.00}{##1}}}
\expandafter\def\csname PY@tok@kp\endcsname{\def\PY@tc##1{\textcolor[rgb]{0.00,0.50,0.00}{##1}}}
\expandafter\def\csname PY@tok@kt\endcsname{\def\PY@tc##1{\textcolor[rgb]{0.69,0.00,0.25}{##1}}}
\expandafter\def\csname PY@tok@o\endcsname{\def\PY@tc##1{\textcolor[rgb]{0.40,0.40,0.40}{##1}}}
\expandafter\def\csname PY@tok@ow\endcsname{\let\PY@bf=\textbf\def\PY@tc##1{\textcolor[rgb]{0.67,0.13,1.00}{##1}}}
\expandafter\def\csname PY@tok@nb\endcsname{\def\PY@tc##1{\textcolor[rgb]{0.00,0.50,0.00}{##1}}}
\expandafter\def\csname PY@tok@nf\endcsname{\def\PY@tc##1{\textcolor[rgb]{0.00,0.00,1.00}{##1}}}
\expandafter\def\csname PY@tok@nc\endcsname{\let\PY@bf=\textbf\def\PY@tc##1{\textcolor[rgb]{0.00,0.00,1.00}{##1}}}
\expandafter\def\csname PY@tok@nn\endcsname{\let\PY@bf=\textbf\def\PY@tc##1{\textcolor[rgb]{0.00,0.00,1.00}{##1}}}
\expandafter\def\csname PY@tok@ne\endcsname{\let\PY@bf=\textbf\def\PY@tc##1{\textcolor[rgb]{0.82,0.25,0.23}{##1}}}
\expandafter\def\csname PY@tok@nv\endcsname{\def\PY@tc##1{\textcolor[rgb]{0.10,0.09,0.49}{##1}}}
\expandafter\def\csname PY@tok@no\endcsname{\def\PY@tc##1{\textcolor[rgb]{0.53,0.00,0.00}{##1}}}
\expandafter\def\csname PY@tok@nl\endcsname{\def\PY@tc##1{\textcolor[rgb]{0.63,0.63,0.00}{##1}}}
\expandafter\def\csname PY@tok@ni\endcsname{\let\PY@bf=\textbf\def\PY@tc##1{\textcolor[rgb]{0.60,0.60,0.60}{##1}}}
\expandafter\def\csname PY@tok@na\endcsname{\def\PY@tc##1{\textcolor[rgb]{0.49,0.56,0.16}{##1}}}
\expandafter\def\csname PY@tok@nt\endcsname{\let\PY@bf=\textbf\def\PY@tc##1{\textcolor[rgb]{0.00,0.50,0.00}{##1}}}
\expandafter\def\csname PY@tok@nd\endcsname{\def\PY@tc##1{\textcolor[rgb]{0.67,0.13,1.00}{##1}}}
\expandafter\def\csname PY@tok@s\endcsname{\def\PY@tc##1{\textcolor[rgb]{0.73,0.13,0.13}{##1}}}
\expandafter\def\csname PY@tok@sd\endcsname{\let\PY@it=\textit\def\PY@tc##1{\textcolor[rgb]{0.73,0.13,0.13}{##1}}}
\expandafter\def\csname PY@tok@si\endcsname{\let\PY@bf=\textbf\def\PY@tc##1{\textcolor[rgb]{0.73,0.40,0.53}{##1}}}
\expandafter\def\csname PY@tok@se\endcsname{\let\PY@bf=\textbf\def\PY@tc##1{\textcolor[rgb]{0.73,0.40,0.13}{##1}}}
\expandafter\def\csname PY@tok@sr\endcsname{\def\PY@tc##1{\textcolor[rgb]{0.73,0.40,0.53}{##1}}}
\expandafter\def\csname PY@tok@ss\endcsname{\def\PY@tc##1{\textcolor[rgb]{0.10,0.09,0.49}{##1}}}
\expandafter\def\csname PY@tok@sx\endcsname{\def\PY@tc##1{\textcolor[rgb]{0.00,0.50,0.00}{##1}}}
\expandafter\def\csname PY@tok@m\endcsname{\def\PY@tc##1{\textcolor[rgb]{0.40,0.40,0.40}{##1}}}
\expandafter\def\csname PY@tok@gh\endcsname{\let\PY@bf=\textbf\def\PY@tc##1{\textcolor[rgb]{0.00,0.00,0.50}{##1}}}
\expandafter\def\csname PY@tok@gu\endcsname{\let\PY@bf=\textbf\def\PY@tc##1{\textcolor[rgb]{0.50,0.00,0.50}{##1}}}
\expandafter\def\csname PY@tok@gd\endcsname{\def\PY@tc##1{\textcolor[rgb]{0.63,0.00,0.00}{##1}}}
\expandafter\def\csname PY@tok@gi\endcsname{\def\PY@tc##1{\textcolor[rgb]{0.00,0.63,0.00}{##1}}}
\expandafter\def\csname PY@tok@gr\endcsname{\def\PY@tc##1{\textcolor[rgb]{1.00,0.00,0.00}{##1}}}
\expandafter\def\csname PY@tok@ge\endcsname{\let\PY@it=\textit}
\expandafter\def\csname PY@tok@gs\endcsname{\let\PY@bf=\textbf}
\expandafter\def\csname PY@tok@gp\endcsname{\let\PY@bf=\textbf\def\PY@tc##1{\textcolor[rgb]{0.00,0.00,0.50}{##1}}}
\expandafter\def\csname PY@tok@go\endcsname{\def\PY@tc##1{\textcolor[rgb]{0.53,0.53,0.53}{##1}}}
\expandafter\def\csname PY@tok@gt\endcsname{\def\PY@tc##1{\textcolor[rgb]{0.00,0.27,0.87}{##1}}}
\expandafter\def\csname PY@tok@err\endcsname{\def\PY@bc##1{\setlength{\fboxsep}{0pt}\fcolorbox[rgb]{1.00,0.00,0.00}{1,1,1}{\strut ##1}}}
\expandafter\def\csname PY@tok@kc\endcsname{\let\PY@bf=\textbf\def\PY@tc##1{\textcolor[rgb]{0.00,0.50,0.00}{##1}}}
\expandafter\def\csname PY@tok@kd\endcsname{\let\PY@bf=\textbf\def\PY@tc##1{\textcolor[rgb]{0.00,0.50,0.00}{##1}}}
\expandafter\def\csname PY@tok@kn\endcsname{\let\PY@bf=\textbf\def\PY@tc##1{\textcolor[rgb]{0.00,0.50,0.00}{##1}}}
\expandafter\def\csname PY@tok@kr\endcsname{\let\PY@bf=\textbf\def\PY@tc##1{\textcolor[rgb]{0.00,0.50,0.00}{##1}}}
\expandafter\def\csname PY@tok@bp\endcsname{\def\PY@tc##1{\textcolor[rgb]{0.00,0.50,0.00}{##1}}}
\expandafter\def\csname PY@tok@fm\endcsname{\def\PY@tc##1{\textcolor[rgb]{0.00,0.00,1.00}{##1}}}
\expandafter\def\csname PY@tok@vc\endcsname{\def\PY@tc##1{\textcolor[rgb]{0.10,0.09,0.49}{##1}}}
\expandafter\def\csname PY@tok@vg\endcsname{\def\PY@tc##1{\textcolor[rgb]{0.10,0.09,0.49}{##1}}}
\expandafter\def\csname PY@tok@vi\endcsname{\def\PY@tc##1{\textcolor[rgb]{0.10,0.09,0.49}{##1}}}
\expandafter\def\csname PY@tok@vm\endcsname{\def\PY@tc##1{\textcolor[rgb]{0.10,0.09,0.49}{##1}}}
\expandafter\def\csname PY@tok@sa\endcsname{\def\PY@tc##1{\textcolor[rgb]{0.73,0.13,0.13}{##1}}}
\expandafter\def\csname PY@tok@sb\endcsname{\def\PY@tc##1{\textcolor[rgb]{0.73,0.13,0.13}{##1}}}
\expandafter\def\csname PY@tok@sc\endcsname{\def\PY@tc##1{\textcolor[rgb]{0.73,0.13,0.13}{##1}}}
\expandafter\def\csname PY@tok@dl\endcsname{\def\PY@tc##1{\textcolor[rgb]{0.73,0.13,0.13}{##1}}}
\expandafter\def\csname PY@tok@s2\endcsname{\def\PY@tc##1{\textcolor[rgb]{0.73,0.13,0.13}{##1}}}
\expandafter\def\csname PY@tok@sh\endcsname{\def\PY@tc##1{\textcolor[rgb]{0.73,0.13,0.13}{##1}}}
\expandafter\def\csname PY@tok@s1\endcsname{\def\PY@tc##1{\textcolor[rgb]{0.73,0.13,0.13}{##1}}}
\expandafter\def\csname PY@tok@mb\endcsname{\def\PY@tc##1{\textcolor[rgb]{0.40,0.40,0.40}{##1}}}
\expandafter\def\csname PY@tok@mf\endcsname{\def\PY@tc##1{\textcolor[rgb]{0.40,0.40,0.40}{##1}}}
\expandafter\def\csname PY@tok@mh\endcsname{\def\PY@tc##1{\textcolor[rgb]{0.40,0.40,0.40}{##1}}}
\expandafter\def\csname PY@tok@mi\endcsname{\def\PY@tc##1{\textcolor[rgb]{0.40,0.40,0.40}{##1}}}
\expandafter\def\csname PY@tok@il\endcsname{\def\PY@tc##1{\textcolor[rgb]{0.40,0.40,0.40}{##1}}}
\expandafter\def\csname PY@tok@mo\endcsname{\def\PY@tc##1{\textcolor[rgb]{0.40,0.40,0.40}{##1}}}
\expandafter\def\csname PY@tok@ch\endcsname{\let\PY@it=\textit\def\PY@tc##1{\textcolor[rgb]{0.25,0.50,0.50}{##1}}}
\expandafter\def\csname PY@tok@cm\endcsname{\let\PY@it=\textit\def\PY@tc##1{\textcolor[rgb]{0.25,0.50,0.50}{##1}}}
\expandafter\def\csname PY@tok@cpf\endcsname{\let\PY@it=\textit\def\PY@tc##1{\textcolor[rgb]{0.25,0.50,0.50}{##1}}}
\expandafter\def\csname PY@tok@c1\endcsname{\let\PY@it=\textit\def\PY@tc##1{\textcolor[rgb]{0.25,0.50,0.50}{##1}}}
\expandafter\def\csname PY@tok@cs\endcsname{\let\PY@it=\textit\def\PY@tc##1{\textcolor[rgb]{0.25,0.50,0.50}{##1}}}

\def\PYZbs{\char`\\}
\def\PYZus{\char`\_}
\def\PYZob{\char`\{}
\def\PYZcb{\char`\}}
\def\PYZca{\char`\^}
\def\PYZam{\char`\&}
\def\PYZlt{\char`\<}
\def\PYZgt{\char`\>}
\def\PYZsh{\char`\#}
\def\PYZpc{\char`\%}
\def\PYZdl{\char`\$}
\def\PYZhy{\char`\-}
\def\PYZsq{\char`\'}
\def\PYZdq{\char`\"}
\def\PYZti{\char`\~}
% for compatibility with earlier versions
\def\PYZat{@}
\def\PYZlb{[}
\def\PYZrb{]}
\makeatother


    % Exact colors from NB
    \definecolor{incolor}{rgb}{0.0, 0.0, 0.5}
    \definecolor{outcolor}{rgb}{0.545, 0.0, 0.0}



    
    % Prevent overflowing lines due to hard-to-break entities
    \sloppy 
    % Setup hyperref package
    \hypersetup{
      breaklinks=true,  % so long urls are correctly broken across lines
      colorlinks=true,
      urlcolor=urlcolor,
      linkcolor=linkcolor,
      citecolor=citecolor,
      }
    % Slightly bigger margins than the latex defaults
    
    \geometry{verbose,tmargin=1in,bmargin=1in,lmargin=1in,rmargin=1in}
    
    

    \begin{document}
    
    
    \maketitle
    
    

    
    \section{python数据可视化之散点图}\label{pythonux6570ux636eux53efux89c6ux5316ux4e4bux6563ux70b9ux56fe}

    \begin{Verbatim}[commandchars=\\\{\}]
{\color{incolor}In [{\color{incolor}2}]:} \PY{k+kn}{import} \PY{n+nn}{numpy} \PY{k}{as} \PY{n+nn}{np}
        \PY{k+kn}{import} \PY{n+nn}{pandas} \PY{k}{as} \PY{n+nn}{pd}
        \PY{k+kn}{import} \PY{n+nn}{matplotlib}\PY{n+nn}{.}\PY{n+nn}{pyplot} \PY{k}{as} \PY{n+nn}{plt}
        \PY{o}{\PYZpc{}}\PY{k}{matplotlib} inline
\end{Verbatim}


    \begin{Verbatim}[commandchars=\\\{\}]
{\color{incolor}In [{\color{incolor}4}]:} \PY{c+c1}{\PYZsh{}读取数据}
        \PY{n}{df} \PY{o}{=} \PY{n}{pd}\PY{o}{.}\PY{n}{read\PYZus{}csv}\PY{p}{(}\PY{l+s+s2}{\PYZdq{}}\PY{l+s+s2}{/Data/mtcars.csv}\PY{l+s+s2}{\PYZdq{}}\PY{p}{)}
        \PY{n}{df}\PY{o}{.}\PY{n}{head}\PY{p}{(}\PY{p}{)}
\end{Verbatim}


\begin{Verbatim}[commandchars=\\\{\}]
{\color{outcolor}Out[{\color{outcolor}4}]:}                 name   mpg  cyl   disp   hp  drat     wt   qsec  vs  am  gear  \textbackslash{}
        0          Mazda RX4  21.0    6  160.0  110  3.90  2.620  16.46   0   1     4   
        1      Mazda RX4 Wag  21.0    6  160.0  110  3.90  2.875  17.02   0   1     4   
        2         Datsun 710  22.8    4  108.0   93  3.85  2.320  18.61   1   1     4   
        3     Hornet 4 Drive  21.4    6  258.0  110  3.08  3.215  19.44   1   0     3   
        4  Hornet Sportabout  18.7    8  360.0  175  3.15  3.440  17.02   0   0     3   
        
           carb  
        0     4  
        1     4  
        2     1  
        3     1  
        4     2  
\end{Verbatim}
            
    \subsection{概念}\label{ux6982ux5ff5}

    什么是散点图?

    反映两组变量每个数据点的值,并且从散点图可以看出它们之间的相关性

    \subsection{散点图的做法}\label{ux6563ux70b9ux56feux7684ux505aux6cd5}

    \begin{Verbatim}[commandchars=\\\{\}]
{\color{incolor}In [{\color{incolor}45}]:} \PY{c+c1}{\PYZsh{}先来做个简单的散点图}
         \PY{n}{plt}\PY{o}{.}\PY{n}{scatter}\PY{p}{(}\PY{n}{df}\PY{p}{[}\PY{l+s+s2}{\PYZdq{}}\PY{l+s+s2}{mpg}\PY{l+s+s2}{\PYZdq{}}\PY{p}{]}\PY{p}{,}\PY{n}{df}\PY{p}{[}\PY{l+s+s2}{\PYZdq{}}\PY{l+s+s2}{wt}\PY{l+s+s2}{\PYZdq{}}\PY{p}{]}\PY{p}{)}
\end{Verbatim}


\begin{Verbatim}[commandchars=\\\{\}]
{\color{outcolor}Out[{\color{outcolor}45}]:} <matplotlib.collections.PathCollection at 0xb4aa518>
\end{Verbatim}
            
    \begin{center}
    \adjustimage{max size={0.9\linewidth}{0.9\paperheight}}{output_7_1.png}
    \end{center}
    { \hspace*{\fill} \\}
    
    接下来,看看完整的散点图如何绘制

    \textbf{面向对象作图(O---O)}

    \begin{Verbatim}[commandchars=\\\{\}]
{\color{incolor}In [{\color{incolor}49}]:} \PY{n}{fig}\PY{p}{,}\PY{n}{ax} \PY{o}{=} \PY{n}{plt}\PY{o}{.}\PY{n}{subplots}\PY{p}{(}\PY{n}{figsize} \PY{o}{=} \PY{p}{(}\PY{l+m+mi}{8}\PY{p}{,}\PY{l+m+mi}{5}\PY{p}{)}\PY{p}{,}\PY{n}{dpi} \PY{o}{=} \PY{l+m+mi}{80}\PY{p}{)} 
         \PY{c+c1}{\PYZsh{}生成一个figure对象和一个坐标系axes对象,并且创建一个图形区域长宽比为8:5,分辨率为每英寸80像素的图}
         
         \PY{n}{ax}\PY{o}{.}\PY{n}{scatter}\PY{p}{(}\PY{n}{df}\PY{p}{[}\PY{l+s+s2}{\PYZdq{}}\PY{l+s+s2}{mpg}\PY{l+s+s2}{\PYZdq{}}\PY{p}{]}\PY{p}{,}\PY{n}{df}\PY{p}{[}\PY{l+s+s2}{\PYZdq{}}\PY{l+s+s2}{wt}\PY{l+s+s2}{\PYZdq{}}\PY{p}{]}\PY{p}{)} \PY{c+c1}{\PYZsh{}作散点图}
         
         \PY{n}{ax}\PY{o}{.}\PY{n}{set\PYZus{}xlabel}\PY{p}{(}\PY{l+s+s2}{\PYZdq{}}\PY{l+s+s2}{mpg}\PY{l+s+s2}{\PYZdq{}}\PY{p}{,}\PY{n}{fontsize}\PY{o}{=}\PY{l+m+mi}{13}\PY{p}{)} \PY{c+c1}{\PYZsh{}添加x轴标签}
         \PY{n}{ax}\PY{o}{.}\PY{n}{set\PYZus{}ylabel}\PY{p}{(}\PY{l+s+s2}{\PYZdq{}}\PY{l+s+s2}{wt}\PY{l+s+s2}{\PYZdq{}}\PY{p}{,}\PY{n}{fontsize}\PY{o}{=}\PY{l+m+mi}{13}\PY{p}{)}  \PY{c+c1}{\PYZsh{}添加y轴标签}
         \PY{n}{ax}\PY{o}{.}\PY{n}{set\PYZus{}title}\PY{p}{(}\PY{l+s+s2}{\PYZdq{}}\PY{l+s+s2}{The relationship of mpg and wt}\PY{l+s+s2}{\PYZdq{}}\PY{p}{,}\PY{n}{fontsize}\PY{o}{=}\PY{l+m+mi}{16}\PY{p}{)} \PY{c+c1}{\PYZsh{}添加主标题}
         \PY{n}{ax}\PY{o}{.}\PY{n}{set\PYZus{}xlim}\PY{p}{(}\PY{l+m+mi}{6}\PY{p}{,}\PY{l+m+mi}{35}\PY{p}{)}  \PY{c+c1}{\PYZsh{}x轴调整取值范围}
         \PY{n}{ax}\PY{o}{.}\PY{n}{set\PYZus{}ylim}\PY{p}{(}\PY{l+m+mi}{1}\PY{p}{,}\PY{l+m+mi}{7}\PY{p}{)}  \PY{c+c1}{\PYZsh{}y轴调整取值范围}
\end{Verbatim}


\begin{Verbatim}[commandchars=\\\{\}]
{\color{outcolor}Out[{\color{outcolor}49}]:} (1, 7)
\end{Verbatim}
            
    \begin{center}
    \adjustimage{max size={0.9\linewidth}{0.9\paperheight}}{output_10_1.png}
    \end{center}
    { \hspace*{\fill} \\}
    
    在解读一张图的时候,可以参照下面的顺序:

\begin{enumerate}
\def\labelenumi{\arabic{enumi}.}
\tightlist
\item
  一句话概括这张图画了什么:``这幅图描绘了汽车的每加仑英里数(mpg)和车重(wt)的关系。''
\item
  说明坐标轴:``图中横轴代表了每加仑英里数(mpg),纵轴代表了车重(wt)。''
\item
  说明图像元素的含义:``每个点对应了一辆车的每加仑英里数和车重,代表了一个样本。''
\item
  深入解读:``我们可以看到,每加仑英里数(mpg)越大,车重(wt)越小,说明每加仑英里数和车重是负相关关系''
\end{enumerate}

    \begin{quote}
散点图的应用场景:查看两个变量间的相关情况,例如房屋平米数与房屋出售价格的关系
\end{quote}

    \subsection{参数介绍}\label{ux53c2ux6570ux4ecbux7ecd}

    ax.scatter(x, y, s=None, c=None, marker=None, cmap=None, norm=None,
vmin=None, vmax=None, alpha=None, linewidths=None, verts=None,
edgecolors=None, *, data=None, **kwargs)

    \begin{Verbatim}[commandchars=\\\{\}]
{\color{incolor}In [{\color{incolor}53}]:} \PY{c+c1}{\PYZsh{}气泡大小由disp变量控制}
         \PY{n}{fig}\PY{p}{,}\PY{n}{ax} \PY{o}{=} \PY{n}{plt}\PY{o}{.}\PY{n}{subplots}\PY{p}{(}\PY{n}{figsize} \PY{o}{=} \PY{p}{(}\PY{l+m+mi}{8}\PY{p}{,}\PY{l+m+mi}{5}\PY{p}{)}\PY{p}{,}\PY{n}{dpi} \PY{o}{=} \PY{l+m+mi}{80}\PY{p}{)}
         
         \PY{n}{ax}\PY{o}{.}\PY{n}{scatter}\PY{p}{(}\PY{n}{df}\PY{p}{[}\PY{l+s+s2}{\PYZdq{}}\PY{l+s+s2}{mpg}\PY{l+s+s2}{\PYZdq{}}\PY{p}{]}\PY{p}{,}\PY{n}{df}\PY{p}{[}\PY{l+s+s2}{\PYZdq{}}\PY{l+s+s2}{wt}\PY{l+s+s2}{\PYZdq{}}\PY{p}{]}\PY{p}{,}\PY{n}{s} \PY{o}{=} \PY{n}{df}\PY{p}{[}\PY{l+s+s2}{\PYZdq{}}\PY{l+s+s2}{disp}\PY{l+s+s2}{\PYZdq{}}\PY{p}{]}\PY{p}{,}\PY{n}{alpha} \PY{o}{=} \PY{l+m+mf}{0.5}\PY{p}{)}    
         \PY{c+c1}{\PYZsh{}df[\PYZdq{}disp\PYZdq{}]作为一个连续型变量}
         
         \PY{n}{ax}\PY{o}{.}\PY{n}{set\PYZus{}xlabel}\PY{p}{(}\PY{l+s+s2}{\PYZdq{}}\PY{l+s+s2}{mpg}\PY{l+s+s2}{\PYZdq{}}\PY{p}{)}
         \PY{n}{ax}\PY{o}{.}\PY{n}{set\PYZus{}ylabel}\PY{p}{(}\PY{l+s+s2}{\PYZdq{}}\PY{l+s+s2}{wt}\PY{l+s+s2}{\PYZdq{}}\PY{p}{)}
         \PY{n}{ax}\PY{o}{.}\PY{n}{set\PYZus{}title}\PY{p}{(}\PY{l+s+s2}{\PYZdq{}}\PY{l+s+s2}{The relationship of mpg and wt}\PY{l+s+s2}{\PYZdq{}}\PY{p}{,}\PY{n}{fontsize}\PY{o}{=}\PY{l+m+mi}{15}\PY{p}{)}
         \PY{n}{ax}\PY{o}{.}\PY{n}{text}\PY{p}{(}\PY{l+m+mi}{25}\PY{p}{,}\PY{l+m+mf}{5.4}\PY{p}{,}\PY{l+s+s2}{\PYZdq{}}\PY{l+s+s2}{bubble size is controlled by }\PY{l+s+s2}{\PYZsq{}}\PY{l+s+s2}{disp}\PY{l+s+s2}{\PYZsq{}}\PY{l+s+s2}{\PYZdq{}}\PY{p}{)}
\end{Verbatim}


\begin{Verbatim}[commandchars=\\\{\}]
{\color{outcolor}Out[{\color{outcolor}53}]:} <matplotlib.text.Text at 0xb8b0908>
\end{Verbatim}
            
    \begin{center}
    \adjustimage{max size={0.9\linewidth}{0.9\paperheight}}{output_15_1.png}
    \end{center}
    { \hspace*{\fill} \\}
    
    这个图相比一般散点图多了一个维度的数据,即将第三个维度的数据大小以气泡大小表示

    应用场景:在研究2个变量的关系时,需要加入连续型变量作为第三维度

    这里涉及到了一个参数s,用于控制散点的大小,如果s输入的是一个固定值,那么就只是单纯的改变散点的大小,不具备维度

    \begin{Verbatim}[commandchars=\\\{\}]
{\color{incolor}In [{\color{incolor}7}]:} \PY{c+c1}{\PYZsh{}设置固定气泡大小}
        \PY{n}{fig}\PY{p}{,}\PY{n}{ax} \PY{o}{=} \PY{n}{plt}\PY{o}{.}\PY{n}{subplots}\PY{p}{(}\PY{n}{figsize} \PY{o}{=} \PY{p}{(}\PY{l+m+mi}{8}\PY{p}{,}\PY{l+m+mi}{5}\PY{p}{)}\PY{p}{,}\PY{n}{dpi} \PY{o}{=} \PY{l+m+mi}{80}\PY{p}{)}
        
        \PY{n}{ax}\PY{o}{.}\PY{n}{scatter}\PY{p}{(}\PY{n}{df}\PY{p}{[}\PY{l+s+s2}{\PYZdq{}}\PY{l+s+s2}{mpg}\PY{l+s+s2}{\PYZdq{}}\PY{p}{]}\PY{p}{,}\PY{n}{df}\PY{p}{[}\PY{l+s+s2}{\PYZdq{}}\PY{l+s+s2}{wt}\PY{l+s+s2}{\PYZdq{}}\PY{p}{]}\PY{p}{,}\PY{n}{s} \PY{o}{=} \PY{l+m+mi}{100}\PY{p}{,}\PY{n}{alpha} \PY{o}{=} \PY{l+m+mf}{0.5}\PY{p}{)}
        
        \PY{n}{ax}\PY{o}{.}\PY{n}{set\PYZus{}xlabel}\PY{p}{(}\PY{l+s+s2}{\PYZdq{}}\PY{l+s+s2}{mpg}\PY{l+s+s2}{\PYZdq{}}\PY{p}{)}
        \PY{n}{ax}\PY{o}{.}\PY{n}{set\PYZus{}ylabel}\PY{p}{(}\PY{l+s+s2}{\PYZdq{}}\PY{l+s+s2}{wt}\PY{l+s+s2}{\PYZdq{}}\PY{p}{)}
        \PY{n}{ax}\PY{o}{.}\PY{n}{set\PYZus{}title}\PY{p}{(}\PY{l+s+s2}{\PYZdq{}}\PY{l+s+s2}{The relationship of mpg and wt}\PY{l+s+s2}{\PYZdq{}}\PY{p}{)}
\end{Verbatim}


\begin{Verbatim}[commandchars=\\\{\}]
{\color{outcolor}Out[{\color{outcolor}7}]:} Text(0.5,1,'The relationship of mpg and wt')
\end{Verbatim}
            
    \begin{center}
    \adjustimage{max size={0.9\linewidth}{0.9\paperheight}}{output_19_1.png}
    \end{center}
    { \hspace*{\fill} \\}
    
    \subsubsection{c
代表点的填充颜色}\label{c-ux4ee3ux8868ux70b9ux7684ux586bux5145ux989cux8272}

    \begin{Verbatim}[commandchars=\\\{\}]
{\color{incolor}In [{\color{incolor}7}]:} \PY{n}{fig}\PY{p}{,}\PY{n}{ax} \PY{o}{=} \PY{n}{plt}\PY{o}{.}\PY{n}{subplots}\PY{p}{(}\PY{p}{)}
        
        \PY{n}{ax}\PY{o}{.}\PY{n}{scatter}\PY{p}{(}\PY{n}{df}\PY{p}{[}\PY{l+s+s2}{\PYZdq{}}\PY{l+s+s2}{mpg}\PY{l+s+s2}{\PYZdq{}}\PY{p}{]}\PY{p}{,}\PY{n}{df}\PY{p}{[}\PY{l+s+s2}{\PYZdq{}}\PY{l+s+s2}{wt}\PY{l+s+s2}{\PYZdq{}}\PY{p}{]}\PY{p}{,}\PY{n}{c}\PY{o}{=}\PY{n}{df}\PY{p}{[}\PY{l+s+s2}{\PYZdq{}}\PY{l+s+s2}{vs}\PY{l+s+s2}{\PYZdq{}}\PY{p}{]}\PY{p}{)}   
        \PY{c+c1}{\PYZsh{}df[\PYZdq{}vs\PYZdq{}]作为一个分类变量}
        
        \PY{n}{ax}\PY{o}{.}\PY{n}{set\PYZus{}xlabel}\PY{p}{(}\PY{l+s+s2}{\PYZdq{}}\PY{l+s+s2}{mpg}\PY{l+s+s2}{\PYZdq{}}\PY{p}{)}
        \PY{n}{ax}\PY{o}{.}\PY{n}{set\PYZus{}ylabel}\PY{p}{(}\PY{l+s+s2}{\PYZdq{}}\PY{l+s+s2}{wt}\PY{l+s+s2}{\PYZdq{}}\PY{p}{)}
        \PY{n}{ax}\PY{o}{.}\PY{n}{set\PYZus{}title}\PY{p}{(}\PY{l+s+s2}{\PYZdq{}}\PY{l+s+s2}{The relationship of mpg and wt}\PY{l+s+s2}{\PYZdq{}}\PY{p}{)}
\end{Verbatim}


\begin{Verbatim}[commandchars=\\\{\}]
{\color{outcolor}Out[{\color{outcolor}7}]:} <matplotlib.text.Text at 0x9fea358>
\end{Verbatim}
            
    \begin{center}
    \adjustimage{max size={0.9\linewidth}{0.9\paperheight}}{output_21_1.png}
    \end{center}
    { \hspace*{\fill} \\}
    
    应用场景:在研究2个变量的关系时,需要加入离散型变量作为第三维度

    但是我们发现好像缺了点什么:我们不知道黄色代表什么值,紫色代表什么值,对吧?

因此我们需要添加1个图例来表明每个颜色代表的是什么数据

    \textbf{思路:已知vs只有0,1两种值,那么我只需要提取出vs==0的数据源和vs==1的数据源,在1张图上用2套数据源画2种颜色的散点图即可}

    \begin{Verbatim}[commandchars=\\\{\}]
{\color{incolor}In [{\color{incolor}60}]:} \PY{c+c1}{\PYZsh{}一种比较容易理解的方法,也是比较死板的做法}
         \PY{n}{x\PYZus{}vs0} \PY{o}{=} \PY{n}{df}\PY{p}{[}\PY{l+s+s2}{\PYZdq{}}\PY{l+s+s2}{mpg}\PY{l+s+s2}{\PYZdq{}}\PY{p}{]}\PY{o}{.}\PY{n}{loc}\PY{p}{[}\PY{n}{df}\PY{p}{[}\PY{l+s+s2}{\PYZdq{}}\PY{l+s+s2}{vs}\PY{l+s+s2}{\PYZdq{}}\PY{p}{]}\PY{o}{==}\PY{l+m+mi}{0}\PY{p}{]}
         \PY{n}{y\PYZus{}vs0} \PY{o}{=} \PY{n}{df}\PY{p}{[}\PY{l+s+s2}{\PYZdq{}}\PY{l+s+s2}{wt}\PY{l+s+s2}{\PYZdq{}}\PY{p}{]}\PY{o}{.}\PY{n}{loc}\PY{p}{[}\PY{n}{df}\PY{p}{[}\PY{l+s+s2}{\PYZdq{}}\PY{l+s+s2}{vs}\PY{l+s+s2}{\PYZdq{}}\PY{p}{]}\PY{o}{==}\PY{l+m+mi}{0}\PY{p}{]}
         \PY{n}{x\PYZus{}vs1} \PY{o}{=} \PY{n}{df}\PY{p}{[}\PY{l+s+s2}{\PYZdq{}}\PY{l+s+s2}{mpg}\PY{l+s+s2}{\PYZdq{}}\PY{p}{]}\PY{o}{.}\PY{n}{loc}\PY{p}{[}\PY{n}{df}\PY{p}{[}\PY{l+s+s2}{\PYZdq{}}\PY{l+s+s2}{vs}\PY{l+s+s2}{\PYZdq{}}\PY{p}{]}\PY{o}{==}\PY{l+m+mi}{1}\PY{p}{]}
         \PY{n}{y\PYZus{}vs1} \PY{o}{=} \PY{n}{df}\PY{p}{[}\PY{l+s+s2}{\PYZdq{}}\PY{l+s+s2}{wt}\PY{l+s+s2}{\PYZdq{}}\PY{p}{]}\PY{o}{.}\PY{n}{loc}\PY{p}{[}\PY{n}{df}\PY{p}{[}\PY{l+s+s2}{\PYZdq{}}\PY{l+s+s2}{vs}\PY{l+s+s2}{\PYZdq{}}\PY{p}{]}\PY{o}{==}\PY{l+m+mi}{1}\PY{p}{]}
         
         \PY{n}{fig}\PY{p}{,}\PY{n}{ax} \PY{o}{=} \PY{n}{plt}\PY{o}{.}\PY{n}{subplots}\PY{p}{(}\PY{p}{)}
         
         \PY{n}{ax}\PY{o}{.}\PY{n}{scatter}\PY{p}{(}\PY{n}{x\PYZus{}vs0}\PY{p}{,}\PY{n}{y\PYZus{}vs0}\PY{p}{,}\PY{n}{c}\PY{o}{=}\PY{l+s+s2}{\PYZdq{}}\PY{l+s+s2}{red}\PY{l+s+s2}{\PYZdq{}}\PY{p}{,}\PY{n}{label}\PY{o}{=}\PY{l+s+s2}{\PYZdq{}}\PY{l+s+s2}{vs=0}\PY{l+s+s2}{\PYZdq{}}\PY{p}{)}
         \PY{n}{ax}\PY{o}{.}\PY{n}{scatter}\PY{p}{(}\PY{n}{x\PYZus{}vs1}\PY{p}{,}\PY{n}{y\PYZus{}vs1}\PY{p}{,}\PY{n}{c}\PY{o}{=}\PY{l+s+s2}{\PYZdq{}}\PY{l+s+s2}{blue}\PY{l+s+s2}{\PYZdq{}}\PY{p}{,}\PY{n}{label}\PY{o}{=}\PY{l+s+s2}{\PYZdq{}}\PY{l+s+s2}{vs=1}\PY{l+s+s2}{\PYZdq{}}\PY{p}{)}
         
         \PY{n}{ax}\PY{o}{.}\PY{n}{set\PYZus{}xlabel}\PY{p}{(}\PY{l+s+s2}{\PYZdq{}}\PY{l+s+s2}{mpg}\PY{l+s+s2}{\PYZdq{}}\PY{p}{)}
         \PY{n}{ax}\PY{o}{.}\PY{n}{set\PYZus{}ylabel}\PY{p}{(}\PY{l+s+s2}{\PYZdq{}}\PY{l+s+s2}{wt}\PY{l+s+s2}{\PYZdq{}}\PY{p}{)}
         \PY{n}{ax}\PY{o}{.}\PY{n}{set\PYZus{}title}\PY{p}{(}\PY{l+s+s2}{\PYZdq{}}\PY{l+s+s2}{The relationship of mpg and wt}\PY{l+s+s2}{\PYZdq{}}\PY{p}{)}
         \PY{n}{ax}\PY{o}{.}\PY{n}{legend}\PY{p}{(}\PY{p}{)}
\end{Verbatim}


\begin{Verbatim}[commandchars=\\\{\}]
{\color{outcolor}Out[{\color{outcolor}60}]:} <matplotlib.legend.Legend at 0xafd2b38>
\end{Verbatim}
            
    \begin{center}
    \adjustimage{max size={0.9\linewidth}{0.9\paperheight}}{output_25_1.png}
    \end{center}
    { \hspace*{\fill} \\}
    
    \textbf{但是有个问题:如果vs有很多个值怎么办,难道要一个个提取出来?所以更智能的做法是:}

    \begin{enumerate}
\def\labelenumi{\arabic{enumi}.}
\tightlist
\item
  得到vs的唯一值
\item
  给每一个值配对一种颜色,创建一个与vs唯一值配对的颜色列表
\item
  将vs唯一值的序列改造成一个带索引的序列,针对这个索引序列写个循环,将vs的每个唯一取值和对应的索引给遍历过去,提取出对应的x和y,根据索引匹配颜色列表,根据序列的值建立标签label,这样就可以智能化地画出带分类图例的散点图了
\end{enumerate}

    第一次循环时,i=0,v=0,x为vs==0时取出的mpg值;y为vs==0时取出的wt值

第二次循环时,i=1,v=1,x为vs==1时取出的mpg值,y为vs==1时取出wt值

    \begin{Verbatim}[commandchars=\\\{\}]
{\color{incolor}In [{\color{incolor}17}]:} \PY{n}{df}\PY{p}{[}\PY{l+s+s1}{\PYZsq{}}\PY{l+s+s1}{vs}\PY{l+s+s1}{\PYZsq{}}\PY{p}{]}\PY{o}{.}\PY{n}{unique}\PY{p}{(}\PY{p}{)}
\end{Verbatim}


\begin{Verbatim}[commandchars=\\\{\}]
{\color{outcolor}Out[{\color{outcolor}17}]:} array([0, 1], dtype=int64)
\end{Verbatim}
            
    \begin{Verbatim}[commandchars=\\\{\}]
{\color{incolor}In [{\color{incolor}9}]:} \PY{c+c1}{\PYZsh{}更智能化的做法}
        \PY{n}{fig}\PY{p}{,}\PY{n}{ax} \PY{o}{=} \PY{n}{plt}\PY{o}{.}\PY{n}{subplots}\PY{p}{(}\PY{p}{)}
        
        \PY{n}{unique} \PY{o}{=} \PY{n+nb}{list}\PY{p}{(}\PY{n+nb}{set}\PY{p}{(}\PY{n}{df}\PY{p}{[}\PY{l+s+s2}{\PYZdq{}}\PY{l+s+s2}{vs}\PY{l+s+s2}{\PYZdq{}}\PY{p}{]}\PY{p}{)}\PY{p}{)}
        \PY{n}{colors} \PY{o}{=} \PY{p}{[}\PY{n}{plt}\PY{o}{.}\PY{n}{cm}\PY{o}{.}\PY{n}{jet}\PY{p}{(}\PY{n}{i}\PY{o}{/}\PY{n+nb}{max}\PY{p}{(}\PY{n}{unique}\PY{p}{)}\PY{o}{*}\PY{l+m+mf}{0.8}\PY{p}{)} \PY{k}{for} \PY{n}{i} \PY{o+ow}{in} \PY{n}{unique}\PY{p}{]}
        \PY{k}{for} \PY{n}{i}\PY{p}{,} \PY{n}{v} \PY{o+ow}{in} \PY{n+nb}{enumerate}\PY{p}{(}\PY{n}{unique}\PY{p}{)}\PY{p}{:}                              \PY{c+c1}{\PYZsh{}将vs唯一值的这个序列改造成一个带索引的序列,并建立一个关于该序列索引和值的循环}
            \PY{n}{x} \PY{o}{=} \PY{p}{[}\PY{n}{df}\PY{p}{[}\PY{l+s+s2}{\PYZdq{}}\PY{l+s+s2}{mpg}\PY{l+s+s2}{\PYZdq{}}\PY{p}{]}\PY{p}{[}\PY{n}{j}\PY{p}{]} \PY{k}{for} \PY{n}{j} \PY{o+ow}{in} \PY{n+nb}{range}\PY{p}{(}\PY{n+nb}{len}\PY{p}{(}\PY{n}{df}\PY{p}{[}\PY{l+s+s2}{\PYZdq{}}\PY{l+s+s2}{mpg}\PY{l+s+s2}{\PYZdq{}}\PY{p}{]}\PY{p}{)}\PY{p}{)} \PY{k}{if} \PY{n}{df}\PY{p}{[}\PY{l+s+s2}{\PYZdq{}}\PY{l+s+s2}{vs}\PY{l+s+s2}{\PYZdq{}}\PY{p}{]}\PY{p}{[}\PY{n}{j}\PY{p}{]} \PY{o}{==} \PY{n}{v}\PY{p}{]} \PY{c+c1}{\PYZsh{}取出对应的mpg}
            \PY{n}{y} \PY{o}{=} \PY{p}{[}\PY{n}{df}\PY{p}{[}\PY{l+s+s2}{\PYZdq{}}\PY{l+s+s2}{wt}\PY{l+s+s2}{\PYZdq{}}\PY{p}{]}\PY{p}{[}\PY{n}{j}\PY{p}{]} \PY{k}{for} \PY{n}{j} \PY{o+ow}{in} \PY{n+nb}{range}\PY{p}{(}\PY{n+nb}{len}\PY{p}{(}\PY{n}{df}\PY{p}{[}\PY{l+s+s2}{\PYZdq{}}\PY{l+s+s2}{wt}\PY{l+s+s2}{\PYZdq{}}\PY{p}{]}\PY{p}{)}\PY{p}{)} \PY{k}{if} \PY{n}{df}\PY{p}{[}\PY{l+s+s2}{\PYZdq{}}\PY{l+s+s2}{vs}\PY{l+s+s2}{\PYZdq{}}\PY{p}{]}\PY{p}{[}\PY{n}{j}\PY{p}{]} \PY{o}{==} \PY{n}{v}\PY{p}{]}   \PY{c+c1}{\PYZsh{}取出对应的wt}
            \PY{n}{ax}\PY{o}{.}\PY{n}{scatter}\PY{p}{(}\PY{n}{x}\PY{p}{,} \PY{n}{y}\PY{p}{,} \PY{n}{c}\PY{o}{=}\PY{n}{colors}\PY{p}{[}\PY{n}{i}\PY{p}{]}\PY{p}{,} \PY{n}{label}\PY{o}{=}\PY{l+s+s2}{\PYZdq{}}\PY{l+s+s2}{vs = }\PY{l+s+s2}{\PYZdq{}}\PY{o}{+} \PY{n+nb}{str}\PY{p}{(}\PY{n}{v}\PY{p}{)}\PY{p}{)}   \PY{c+c1}{\PYZsh{}根据取出的mpg和wt、i匹配的颜色、以及v对应的标签来画散点图}
        
        \PY{n}{ax}\PY{o}{.}\PY{n}{set\PYZus{}xlabel}\PY{p}{(}\PY{l+s+s2}{\PYZdq{}}\PY{l+s+s2}{mpg}\PY{l+s+s2}{\PYZdq{}}\PY{p}{)}
        \PY{n}{ax}\PY{o}{.}\PY{n}{set\PYZus{}ylabel}\PY{p}{(}\PY{l+s+s2}{\PYZdq{}}\PY{l+s+s2}{wt}\PY{l+s+s2}{\PYZdq{}}\PY{p}{)}
        \PY{n}{ax}\PY{o}{.}\PY{n}{set\PYZus{}title}\PY{p}{(}\PY{l+s+s2}{\PYZdq{}}\PY{l+s+s2}{The relationship of mpg and wt}\PY{l+s+s2}{\PYZdq{}}\PY{p}{)}
        \PY{n}{ax}\PY{o}{.}\PY{n}{legend}\PY{p}{(}\PY{p}{)}
\end{Verbatim}


\begin{Verbatim}[commandchars=\\\{\}]
{\color{outcolor}Out[{\color{outcolor}9}]:} <matplotlib.legend.Legend at 0x1575cf0>
\end{Verbatim}
            
    \begin{center}
    \adjustimage{max size={0.9\linewidth}{0.9\paperheight}}{output_30_1.png}
    \end{center}
    { \hspace*{\fill} \\}
    
    \begin{Verbatim}[commandchars=\\\{\}]
{\color{incolor}In [{\color{incolor}10}]:} \PY{n}{unique}
\end{Verbatim}


\begin{Verbatim}[commandchars=\\\{\}]
{\color{outcolor}Out[{\color{outcolor}10}]:} [0, 1]
\end{Verbatim}
            
    在上面的例子里涉及到了一个参数c,用于点的颜色设置,如果c输入的是一个固定值,那么就只是单纯的改变散点的颜色,不具备维度

    \begin{Verbatim}[commandchars=\\\{\}]
{\color{incolor}In [{\color{incolor}32}]:} \PY{n}{fig}\PY{p}{,}\PY{n}{ax} \PY{o}{=} \PY{n}{plt}\PY{o}{.}\PY{n}{subplots}\PY{p}{(}\PY{p}{)}
         
         \PY{n}{ax}\PY{o}{.}\PY{n}{scatter}\PY{p}{(}\PY{n}{df}\PY{p}{[}\PY{l+s+s2}{\PYZdq{}}\PY{l+s+s2}{mpg}\PY{l+s+s2}{\PYZdq{}}\PY{p}{]}\PY{p}{,}\PY{n}{df}\PY{p}{[}\PY{l+s+s2}{\PYZdq{}}\PY{l+s+s2}{wt}\PY{l+s+s2}{\PYZdq{}}\PY{p}{]}\PY{p}{,}\PY{n}{c} \PY{o}{=} \PY{l+s+s2}{\PYZdq{}}\PY{l+s+s2}{red}\PY{l+s+s2}{\PYZdq{}}\PY{p}{)}
         
         \PY{n}{ax}\PY{o}{.}\PY{n}{set\PYZus{}xlabel}\PY{p}{(}\PY{l+s+s2}{\PYZdq{}}\PY{l+s+s2}{mpg}\PY{l+s+s2}{\PYZdq{}}\PY{p}{)}
         \PY{n}{ax}\PY{o}{.}\PY{n}{set\PYZus{}ylabel}\PY{p}{(}\PY{l+s+s2}{\PYZdq{}}\PY{l+s+s2}{wt}\PY{l+s+s2}{\PYZdq{}}\PY{p}{)}
         \PY{n}{ax}\PY{o}{.}\PY{n}{set\PYZus{}title}\PY{p}{(}\PY{l+s+s2}{\PYZdq{}}\PY{l+s+s2}{The relationship of mpg and wt}\PY{l+s+s2}{\PYZdq{}}\PY{p}{)}
\end{Verbatim}


\begin{Verbatim}[commandchars=\\\{\}]
{\color{outcolor}Out[{\color{outcolor}32}]:} <matplotlib.text.Text at 0x14f98fd0>
\end{Verbatim}
            
    \begin{center}
    \adjustimage{max size={0.9\linewidth}{0.9\paperheight}}{output_33_1.png}
    \end{center}
    { \hspace*{\fill} \\}
    
    如果你希望设置一些自己想要的颜色,可以用RGB的方式来调色

    \begin{Verbatim}[commandchars=\\\{\}]
{\color{incolor}In [{\color{incolor}64}]:} \PY{n}{fig}\PY{p}{,}\PY{n}{ax} \PY{o}{=} \PY{n}{plt}\PY{o}{.}\PY{n}{subplots}\PY{p}{(}\PY{p}{)}
         
         \PY{n}{ax}\PY{o}{.}\PY{n}{scatter}\PY{p}{(}\PY{n}{df}\PY{p}{[}\PY{l+s+s2}{\PYZdq{}}\PY{l+s+s2}{mpg}\PY{l+s+s2}{\PYZdq{}}\PY{p}{]}\PY{p}{,}\PY{n}{df}\PY{p}{[}\PY{l+s+s2}{\PYZdq{}}\PY{l+s+s2}{wt}\PY{l+s+s2}{\PYZdq{}}\PY{p}{]}\PY{p}{,}\PY{n}{c} \PY{o}{=} \PY{p}{(}\PY{l+m+mi}{59}\PY{o}{/}\PY{l+m+mi}{255}\PY{p}{,}\PY{l+m+mi}{108}\PY{o}{/}\PY{l+m+mi}{255}\PY{p}{,}\PY{l+m+mi}{199}\PY{o}{/}\PY{l+m+mi}{255}\PY{p}{)}\PY{p}{)}
         
         \PY{n}{ax}\PY{o}{.}\PY{n}{set\PYZus{}xlabel}\PY{p}{(}\PY{l+s+s2}{\PYZdq{}}\PY{l+s+s2}{mpg}\PY{l+s+s2}{\PYZdq{}}\PY{p}{)}
         \PY{n}{ax}\PY{o}{.}\PY{n}{set\PYZus{}ylabel}\PY{p}{(}\PY{l+s+s2}{\PYZdq{}}\PY{l+s+s2}{wt}\PY{l+s+s2}{\PYZdq{}}\PY{p}{)}
         \PY{n}{ax}\PY{o}{.}\PY{n}{set\PYZus{}title}\PY{p}{(}\PY{l+s+s2}{\PYZdq{}}\PY{l+s+s2}{The relationship of mpg and wt}\PY{l+s+s2}{\PYZdq{}}\PY{p}{)}
\end{Verbatim}


\begin{Verbatim}[commandchars=\\\{\}]
{\color{outcolor}Out[{\color{outcolor}64}]:} <matplotlib.text.Text at 0xb4aa438>
\end{Verbatim}
            
    \begin{center}
    \adjustimage{max size={0.9\linewidth}{0.9\paperheight}}{output_35_1.png}
    \end{center}
    { \hspace*{\fill} \\}
    
    \subsubsection{marker表示
点的形状}\label{markerux8868ux793a-ux70b9ux7684ux5f62ux72b6}

    \begin{Verbatim}[commandchars=\\\{\}]
{\color{incolor}In [{\color{incolor}69}]:} \PY{c+c1}{\PYZsh{}设置点的形状}
         \PY{n}{fig}\PY{p}{,}\PY{n}{ax} \PY{o}{=} \PY{n}{plt}\PY{o}{.}\PY{n}{subplots}\PY{p}{(}\PY{n}{figsize} \PY{o}{=} \PY{p}{(}\PY{l+m+mi}{8}\PY{p}{,}\PY{l+m+mi}{5}\PY{p}{)}\PY{p}{,}\PY{n}{dpi} \PY{o}{=} \PY{l+m+mi}{80}\PY{p}{)}
         
         \PY{n}{ax}\PY{o}{.}\PY{n}{scatter}\PY{p}{(}\PY{n}{df}\PY{p}{[}\PY{l+s+s2}{\PYZdq{}}\PY{l+s+s2}{mpg}\PY{l+s+s2}{\PYZdq{}}\PY{p}{]}\PY{p}{,}\PY{n}{df}\PY{p}{[}\PY{l+s+s2}{\PYZdq{}}\PY{l+s+s2}{wt}\PY{l+s+s2}{\PYZdq{}}\PY{p}{]}\PY{p}{,}\PY{n}{marker}\PY{o}{=}\PY{l+s+s2}{\PYZdq{}}\PY{l+s+s2}{+}\PY{l+s+s2}{\PYZdq{}}\PY{p}{)}
         
         \PY{n}{ax}\PY{o}{.}\PY{n}{set\PYZus{}xlabel}\PY{p}{(}\PY{l+s+s2}{\PYZdq{}}\PY{l+s+s2}{mpg}\PY{l+s+s2}{\PYZdq{}}\PY{p}{)}
         \PY{n}{ax}\PY{o}{.}\PY{n}{set\PYZus{}ylabel}\PY{p}{(}\PY{l+s+s2}{\PYZdq{}}\PY{l+s+s2}{wt}\PY{l+s+s2}{\PYZdq{}}\PY{p}{)}
         \PY{n}{ax}\PY{o}{.}\PY{n}{set\PYZus{}title}\PY{p}{(}\PY{l+s+s2}{\PYZdq{}}\PY{l+s+s2}{The relationship of mpg and wt}\PY{l+s+s2}{\PYZdq{}}\PY{p}{)}
\end{Verbatim}


\begin{Verbatim}[commandchars=\\\{\}]
{\color{outcolor}Out[{\color{outcolor}69}]:} <matplotlib.text.Text at 0xbb87320>
\end{Verbatim}
            
    \begin{center}
    \adjustimage{max size={0.9\linewidth}{0.9\paperheight}}{output_37_1.png}
    \end{center}
    { \hspace*{\fill} \\}
    
    \subsubsection{edgecolors表示点的边界颜色}\label{edgecolorsux8868ux793aux70b9ux7684ux8fb9ux754cux989cux8272}

    \begin{Verbatim}[commandchars=\\\{\}]
{\color{incolor}In [{\color{incolor}16}]:} \PY{n}{fig}\PY{p}{,}\PY{n}{ax} \PY{o}{=} \PY{n}{plt}\PY{o}{.}\PY{n}{subplots}\PY{p}{(}\PY{n}{figsize} \PY{o}{=} \PY{p}{(}\PY{l+m+mi}{8}\PY{p}{,}\PY{l+m+mi}{5}\PY{p}{)}\PY{p}{,}\PY{n}{dpi} \PY{o}{=} \PY{l+m+mi}{80}\PY{p}{)}
         
         \PY{n}{ax}\PY{o}{.}\PY{n}{scatter}\PY{p}{(}\PY{n}{df}\PY{p}{[}\PY{l+s+s2}{\PYZdq{}}\PY{l+s+s2}{mpg}\PY{l+s+s2}{\PYZdq{}}\PY{p}{]}\PY{p}{,}\PY{n}{df}\PY{p}{[}\PY{l+s+s2}{\PYZdq{}}\PY{l+s+s2}{wt}\PY{l+s+s2}{\PYZdq{}}\PY{p}{]}\PY{p}{,}\PY{n}{c}\PY{o}{=}\PY{l+s+s2}{\PYZdq{}}\PY{l+s+s2}{white}\PY{l+s+s2}{\PYZdq{}}\PY{p}{,}\PY{n}{edgecolors}\PY{o}{=}\PY{p}{(}\PY{l+m+mi}{244}\PY{o}{/}\PY{l+m+mi}{255}\PY{p}{,}\PY{l+m+mi}{119}\PY{o}{/}\PY{l+m+mi}{255}\PY{p}{,}\PY{l+m+mi}{38}\PY{o}{/}\PY{l+m+mi}{255}\PY{p}{)}\PY{p}{)}
         
         \PY{n}{ax}\PY{o}{.}\PY{n}{set\PYZus{}xlabel}\PY{p}{(}\PY{l+s+s2}{\PYZdq{}}\PY{l+s+s2}{mpg}\PY{l+s+s2}{\PYZdq{}}\PY{p}{)}
         \PY{n}{ax}\PY{o}{.}\PY{n}{set\PYZus{}ylabel}\PY{p}{(}\PY{l+s+s2}{\PYZdq{}}\PY{l+s+s2}{wt}\PY{l+s+s2}{\PYZdq{}}\PY{p}{)}
         \PY{n}{ax}\PY{o}{.}\PY{n}{set\PYZus{}title}\PY{p}{(}\PY{l+s+s2}{\PYZdq{}}\PY{l+s+s2}{The relationship of mpg and wt}\PY{l+s+s2}{\PYZdq{}}\PY{p}{)}
\end{Verbatim}


\begin{Verbatim}[commandchars=\\\{\}]
{\color{outcolor}Out[{\color{outcolor}16}]:} <matplotlib.text.Text at 0xb36cef0>
\end{Verbatim}
            
    \begin{center}
    \adjustimage{max size={0.9\linewidth}{0.9\paperheight}}{output_39_1.png}
    \end{center}
    { \hspace*{\fill} \\}
    
    \subsection{画多个散点图}\label{ux753bux591aux4e2aux6563ux70b9ux56fe}

    \subsubsection{在一张图里画多个散点图}\label{ux5728ux4e00ux5f20ux56feux91ccux753bux591aux4e2aux6563ux70b9ux56fe}

    \begin{Verbatim}[commandchars=\\\{\}]
{\color{incolor}In [{\color{incolor}19}]:} \PY{n}{fig} \PY{o}{=} \PY{n}{plt}\PY{o}{.}\PY{n}{figure}\PY{p}{(}\PY{n}{figsize}\PY{o}{=}\PY{p}{(}\PY{l+m+mi}{8}\PY{p}{,}\PY{l+m+mi}{5}\PY{p}{)}\PY{p}{,}\PY{n}{dpi}\PY{o}{=}\PY{l+m+mi}{80}\PY{p}{)}
         \PY{n}{ax1} \PY{o}{=} \PY{n}{fig}\PY{o}{.}\PY{n}{add\PYZus{}subplot}\PY{p}{(}\PY{l+m+mi}{221}\PY{p}{)}
         \PY{n}{ax2} \PY{o}{=} \PY{n}{fig}\PY{o}{.}\PY{n}{add\PYZus{}subplot}\PY{p}{(}\PY{l+m+mi}{224}\PY{p}{)}
         
         \PY{n}{ax1}\PY{o}{.}\PY{n}{scatter}\PY{p}{(}\PY{n}{df}\PY{p}{[}\PY{l+s+s2}{\PYZdq{}}\PY{l+s+s2}{mpg}\PY{l+s+s2}{\PYZdq{}}\PY{p}{]}\PY{p}{,}\PY{n}{df}\PY{p}{[}\PY{l+s+s2}{\PYZdq{}}\PY{l+s+s2}{wt}\PY{l+s+s2}{\PYZdq{}}\PY{p}{]}\PY{p}{)}
         \PY{n}{ax1}\PY{o}{.}\PY{n}{set\PYZus{}xlabel}\PY{p}{(}\PY{l+s+s2}{\PYZdq{}}\PY{l+s+s2}{mpg}\PY{l+s+s2}{\PYZdq{}}\PY{p}{)}
         \PY{n}{ax1}\PY{o}{.}\PY{n}{set\PYZus{}ylabel}\PY{p}{(}\PY{l+s+s2}{\PYZdq{}}\PY{l+s+s2}{wt}\PY{l+s+s2}{\PYZdq{}}\PY{p}{)}
         \PY{n}{ax1}\PY{o}{.}\PY{n}{set\PYZus{}title}\PY{p}{(}\PY{l+s+s2}{\PYZdq{}}\PY{l+s+s2}{The relationship of mpg and wt}\PY{l+s+s2}{\PYZdq{}}\PY{p}{)}
         
         \PY{n}{ax2}\PY{o}{.}\PY{n}{scatter}\PY{p}{(}\PY{n}{df}\PY{p}{[}\PY{l+s+s2}{\PYZdq{}}\PY{l+s+s2}{mpg}\PY{l+s+s2}{\PYZdq{}}\PY{p}{]}\PY{p}{,}\PY{n}{df}\PY{p}{[}\PY{l+s+s2}{\PYZdq{}}\PY{l+s+s2}{qsec}\PY{l+s+s2}{\PYZdq{}}\PY{p}{]}\PY{p}{)}
         \PY{n}{ax2}\PY{o}{.}\PY{n}{set\PYZus{}xlabel}\PY{p}{(}\PY{l+s+s2}{\PYZdq{}}\PY{l+s+s2}{mpg}\PY{l+s+s2}{\PYZdq{}}\PY{p}{)}
         \PY{n}{ax2}\PY{o}{.}\PY{n}{set\PYZus{}ylabel}\PY{p}{(}\PY{l+s+s2}{\PYZdq{}}\PY{l+s+s2}{qsec}\PY{l+s+s2}{\PYZdq{}}\PY{p}{)}
         \PY{n}{ax2}\PY{o}{.}\PY{n}{set\PYZus{}title}\PY{p}{(}\PY{l+s+s2}{\PYZdq{}}\PY{l+s+s2}{The relationship of mpg and qsec}\PY{l+s+s2}{\PYZdq{}}\PY{p}{)}
\end{Verbatim}


\begin{Verbatim}[commandchars=\\\{\}]
{\color{outcolor}Out[{\color{outcolor}19}]:} Text(0.5,1,'The relationship of mpg and qsec')
\end{Verbatim}
            
    \begin{center}
    \adjustimage{max size={0.9\linewidth}{0.9\paperheight}}{output_42_1.png}
    \end{center}
    { \hspace*{\fill} \\}
    
    \subsubsection{在多张图里画散点图}\label{ux5728ux591aux5f20ux56feux91ccux753bux6563ux70b9ux56fe}

    \begin{Verbatim}[commandchars=\\\{\}]
{\color{incolor}In [{\color{incolor}20}]:} \PY{c+c1}{\PYZsh{}生成第一张图}
         \PY{n}{fig1} \PY{o}{=} \PY{n}{plt}\PY{o}{.}\PY{n}{figure}\PY{p}{(}\PY{p}{)}
         
         \PY{n}{ax1} \PY{o}{=} \PY{n}{fig1}\PY{o}{.}\PY{n}{add\PYZus{}subplot}\PY{p}{(}\PY{l+m+mi}{111}\PY{p}{)}
         
         \PY{n}{ax1}\PY{o}{.}\PY{n}{scatter}\PY{p}{(}\PY{n}{df}\PY{p}{[}\PY{l+s+s2}{\PYZdq{}}\PY{l+s+s2}{mpg}\PY{l+s+s2}{\PYZdq{}}\PY{p}{]}\PY{p}{,}\PY{n}{df}\PY{p}{[}\PY{l+s+s2}{\PYZdq{}}\PY{l+s+s2}{wt}\PY{l+s+s2}{\PYZdq{}}\PY{p}{]}\PY{p}{)}
         
         \PY{n}{ax1}\PY{o}{.}\PY{n}{set\PYZus{}xlabel}\PY{p}{(}\PY{l+s+s2}{\PYZdq{}}\PY{l+s+s2}{mpg}\PY{l+s+s2}{\PYZdq{}}\PY{p}{)}
         \PY{n}{ax1}\PY{o}{.}\PY{n}{set\PYZus{}ylabel}\PY{p}{(}\PY{l+s+s2}{\PYZdq{}}\PY{l+s+s2}{wt}\PY{l+s+s2}{\PYZdq{}}\PY{p}{)}
         \PY{n}{ax1}\PY{o}{.}\PY{n}{set\PYZus{}title}\PY{p}{(}\PY{l+s+s2}{\PYZdq{}}\PY{l+s+s2}{The relationship of mpg and wt}\PY{l+s+s2}{\PYZdq{}}\PY{p}{)}
         
         \PY{c+c1}{\PYZsh{}生成第二张图}
         \PY{n}{fig2} \PY{o}{=} \PY{n}{plt}\PY{o}{.}\PY{n}{figure}\PY{p}{(}\PY{p}{)}
         
         \PY{n}{ax2} \PY{o}{=} \PY{n}{fig2}\PY{o}{.}\PY{n}{add\PYZus{}subplot}\PY{p}{(}\PY{l+m+mi}{111}\PY{p}{)}
         
         \PY{n}{ax2}\PY{o}{.}\PY{n}{scatter}\PY{p}{(}\PY{n}{df}\PY{p}{[}\PY{l+s+s2}{\PYZdq{}}\PY{l+s+s2}{mpg}\PY{l+s+s2}{\PYZdq{}}\PY{p}{]}\PY{p}{,}\PY{n}{df}\PY{p}{[}\PY{l+s+s2}{\PYZdq{}}\PY{l+s+s2}{qsec}\PY{l+s+s2}{\PYZdq{}}\PY{p}{]}\PY{p}{)}
         
         \PY{n}{ax2}\PY{o}{.}\PY{n}{set\PYZus{}xlabel}\PY{p}{(}\PY{l+s+s2}{\PYZdq{}}\PY{l+s+s2}{mpg}\PY{l+s+s2}{\PYZdq{}}\PY{p}{)}
         \PY{n}{ax2}\PY{o}{.}\PY{n}{set\PYZus{}ylabel}\PY{p}{(}\PY{l+s+s2}{\PYZdq{}}\PY{l+s+s2}{qsec}\PY{l+s+s2}{\PYZdq{}}\PY{p}{)}
         \PY{n}{ax2}\PY{o}{.}\PY{n}{set\PYZus{}title}\PY{p}{(}\PY{l+s+s2}{\PYZdq{}}\PY{l+s+s2}{The relationship of mpg and qsec}\PY{l+s+s2}{\PYZdq{}}\PY{p}{)}
\end{Verbatim}


\begin{Verbatim}[commandchars=\\\{\}]
{\color{outcolor}Out[{\color{outcolor}20}]:} Text(0.5,1,'The relationship of mpg and qsec')
\end{Verbatim}
            
    \begin{center}
    \adjustimage{max size={0.9\linewidth}{0.9\paperheight}}{output_44_1.png}
    \end{center}
    { \hspace*{\fill} \\}
    
    \begin{center}
    \adjustimage{max size={0.9\linewidth}{0.9\paperheight}}{output_44_2.png}
    \end{center}
    { \hspace*{\fill} \\}
    

    % Add a bibliography block to the postdoc
    
    
    
    \end{document}
